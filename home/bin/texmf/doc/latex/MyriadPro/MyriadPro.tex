\documentclass[paper=a4,% Papierformat
12pt,% Schriftgr��e
DIV=10,% Layout
]{scrartcl} % Die Klasse
\usepackage[latin1]{inputenc}% Eingabekodierung
\usepackage[T1]{fontenc}% Schriftkodierung
\usepackage{textcomp,booktabs,calc}

%% Schriften
\usepackage[lf,default,semibold]{MyriadPro}% Adobe Myriad Pro
\linespread{1.06}% Zeilenabstand
\normalfont\normalsize
\typearea[current]{current}% Neuberechnung des Satzspiegels

%% Infos und Lesezeichen im PDF
\usepackage[pdfpagelabels=true]{hyperref}
\hypersetup{%
  plainpages=false,
  pdfpagemode=UseNone,
  bookmarksnumbered=true,
  pdfborder={0 0 0},
  pdftitle={LaTeX-Support for Adobe Myriad Pro},
  pdfauthor={Michael G�hrken},
}

%% Titelei
\title{\LaTeX-Support for Adobe Myriad Pro}
\author{Michael G�hrken}
\date{v1.0 -- \today}

\newcommand*\pkg[1]{\mbox{\textsf{#1}}}

\makeatletter
\newcommand*\option{\@ifstar\option@default\option@}
\newcommand*\option@default[1]{\option@{#1}*}
\newcommand*\option@[1]{\textsf{#1}}
\makeatother

%\setlength{\leftmargini}{1em}
%\setlength{\parindent}{1em}
\newcommand\tabindent{\noindent\hspace{\leftmargini}}
\newlength\optionswidth
\setlength{\optionswidth}{8em}
\providecommand\newblock{}
\newenvironment{options}[1][{\makebox[\optionswidth]{}}]{%
  \list{}{%
    \setlength{\labelwidth}{\optionswidth}%
    \setlength{\labelsep}{2\tabcolsep}%
    \setlength{\leftmargin}{\leftmargini+\labelwidth+\labelsep}%
    \setlength{\rightmargin}{0pt}%
    \setlength{\topsep}{\medskipamount}%
    \setlength{\parsep}{0pt}%
    \setlength{\itemsep}{0pt}%
    \renewcommand*\makelabel[1]{%
      \parbox[t]{\labelwidth}{\raggedright\hspace{0pt}##1}}%
    \renewcommand\newblock{\medskip}%
    \raggedright
  }%
}{%
  \endlist
}

\begin{document}
\maketitle

\section{Overview}

The \pkg{MyriadPro} package provides support for the Adobe Myriad Pro font family from Adobe. You can use these fonts in a \LaTeX\ document by adding the command
\begin{quote}
  \verb|\usepackage{MyriadPro}|
\end{quote}
to the preamble. This will change the sans-serif font to Adobe Myriad Pro.


The \pkg{MyriadPro} package automatically loads the \pkg{textcomp} package and the \pkg{fontaxes} package\footnote{The \pkg{fontaxes} package ist part of the MinionPro distribution, available at \url{http://www.ctan.org/tex-archive/fonts/minionpro/}}. If you want to pass options to this package you can either 
put the corresponding \verb|\usepackage| command before the \verb|\usepackage{MyriadPro}| or 
you can include the options in the \verb|\documentclass| command.

\section{Options}

\subsection{Text font}

By default MyriadPro not used as the text font but as the sans-serif font. To use the sans-serif font as the default text font, the following options is required. It is also possible to use the semibold instead of the bold weight for commands like \verb|\textbf{}| and \verb|\bfseries|.

\begin{options}
  \item[\option{default}]        use sans-serif font as default text font
  \item[\option{semibold}]        use semibold instead of bold weight
\end{options}

\subsection{Figure selection}

Adobe Myriad Pro offers four different figure versions. A detailed description is given in Section~\ref{sec:fig}. The default version can be selected by the following options:

\begin{options}
  \item[\option*{osf}]        use text figures
  \item[\option{lf}]          use lining figures
 \newblock
  \item[\option*{pr}]        use proportional figures
  \item[\option{tab}]          use tabular figures
\end{options}


\section{Figure selection}
\label{sec:fig}

Adobe Myriad Pro offers four different figure versions. One can choose between \emph{text figures} (lowercase figures) and \emph{lining figures} (uppercase figures) and one can choose between \emph{proportional} figures (figures with different widths) and \emph{tabular} figures (all figures have the same width, useful mainly for tables).

\medskip\tabindent
\begin{tabular}{@{}lll@{}}
  \toprule
                & text figures & lining figures \\
  \midrule
   proportional & \figureversion{text,proportional}0123456789
                & \figureversion{lining,proportional}0123456789 \\
   tabular      & \figureversion{text,tabular}0123456789
                & \figureversion{lining,tabular}0123456789 \\
  \bottomrule
\end{tabular}

\medskip\noindent
The \verb|\figureversion| command can be used to switch between different figure versions. Possible parameters are:

\begin{options}
  \item[\option{text}, \option{osf}]          text figures
  \item[\option{lining}, \option{lf}]         lining figures
  \item[\option{tabular}, \option{tab}]       tabular figures
  \item[\option{proportional}, \option{prop}] proportional figures
\end{options}

 \section{NFSS classification}

 Parenthesised combinations are provided via substitutions.

 \nobreak\medskip\noindent
 \begingroup
 \centering\let\>=\\%
 \begin{tabular}{@{}p{7.5em}p{7em}lp{9em}@{}}
   \toprule
   encoding & family & series & shape \\
   \midrule
   \raggedright T1, TS1, LY1,&
   MyriadPro-OsF, MyriadPro-LF, MyriadPro-TOsF, MyriadPro-TLF &
   l, m, sb, b (bx) &
   n, it (sl) \\
   \bottomrule
 \end{tabular}\par
 \endgroup

\section{Further information}

For further information about the provided commands, please refer to the documentation included with the \pkg{MinionPro} package\footnote{Available at \url{http://www.ctan.org/tex-archive/fonts/minionpro/}}.


\end{document}