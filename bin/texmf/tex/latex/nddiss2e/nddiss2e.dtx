% \iffalse meta-comment
%
% Copyright (C) 2004-2005 by Sameer Vijay
%
% This file may be distributed and/or modified under the
% conditions of the LaTeX Project Public License, either
% version 1.2 of this license or (at your option) any later
% version. The latest version of this license is in
%
%    http://www.latex-project.org/lppl.txt
%
%
% \fi
%
%
%  \CheckSum{1260}
%  \CharacterTable
%   {Upper-case    \A\B\C\D\E\F\G\H\I\J\K\L\M\N\O\P\Q\R\S\T\U\V\W\X\Y\Z
%    Lower-case    \a\b\c\d\e\f\g\h\i\j\k\l\m\n\o\p\q\r\s\t\u\v\w\x\y\z
%    Digits        \0\1\2\3\4\5\6\7\8\9
%    Exclamation   \!     Double quote  \"     Hash (number) \#
%    Dollar        \$     Percent       \%     Ampersand     \&
%    Acute accent  \'     Left paren    \(     Right paren   \)
%    Asterisk      \*     Plus          \+     Comma         \,
%    Minus         \-     Point         \.     Solidus       \/
%    Colon         \:     Semicolon     \;     Less than     \<
%    Equals        \=     Greater than  \>     Question mark \?
%    Commercial at \@     Left bracket  \[     Backslash     \\
%    Right bracket \]     Circumflex    \^     Underscore    \_
%    Grave accent  \`     Left brace    \{     Vertical bar  \|
%    Right brace   \}     Tilde         \~}
%
% \changes{v0.98}{2004/04/15}{Initial {\em beta} version}
% \renewcommand{\generalname}{Release}
% \changes{v1.0}{2004/06/15}{First release}
% \renewcommand{\generalname}{General}
% \changes{v1.1}{2005/04/14}{Minor changes and clean-up}
% \changes{v2.0}{2005/05/14}{Some bugfixes, cleaned some of documentation}
% \changes{v2.1}{2005/06/14}{More bugfixes, changes in documentation}
% \renewcommand{\generalname}{Release}
% \changes{v3.0}{2005/07/27}{Major revamp and clean-up of the code, added
% \texttt{numrefs} and \texttt{textrefs} to allow different kinds of citation
% styles, added some more macros and modified others,
% changed the titlepage a bit, completed source documentation}
%
%
% \iffalse (Don't show this in the doc).
%    \begin{macrocode}
%<*driver>
\ProvidesFile{nddiss2e.dtx}
\documentclass{ltxdoc}
\setcounter{secnumdepth}{2}
\setcounter{tocdepth}{2}
\DeclareRobustCommand{\nddiss}{%
         \textsf{{\scshape nd}diss}\kern-0.03em%
         2$_\mathsf{\textstyle\varepsilon}$}
\providecommand{\dissfileversion}{3.0}
\providecommand{\dissfiledate}{2005/07/27}
\DisableCrossrefs
\CodelineNumbered
\RecordChanges
\begin{document}
    \DocInput{nddiss2e.dtx}
\end{document}
%</driver>
%    \end{macrocode}
% \fi
%
% \title{The \nddiss\/ class\thanks{Version~\dissfileversion,%
%        dated~\dissfiledate.}}
%
% \author{Sameer Vijay\thanks{Inspiration from an
% earlier {\sffamily NDThesis} class by D. A. Peterson}\\
% }
% \date{\dissfiledate}
%
% \maketitle
%
% \begin{abstract}
% This \nddiss\space class is based on the standard \LaTeXe\/ \textsf{book}
% class and is an extensive rewrite of the earlier \textsf{NDthesis} class
% file, incorporating changes for \LaTeXe\space  and pdf\LaTeX\space as well
% as many other improvements.  This class conforms with the requirements of the
% Graduate School guidelines published in Spring 2004 for the layout of the Ph.D.
% dissertations and Master's theses.  In reading this documentation you will
% find that I assume that the
% reader has working knowledge of \LaTeXe\/.
% \end{abstract}
%
% \tableofcontents
%
% \section{Introduction}
% \label{sec:intro}
% This document describes the \LaTeXe\space document class \nddiss,
% suitable\footnote{In my opinion, but with no guarantee that you or other users
% will agree. I shall not be liable for any consequence, good or bad,
% of anyone's use of this software.} for producing dissertations and
% theses according to the Spring 2004 guidelines of the Graduate School at the
% University of Notre Dame. The latest version of this class and
% related documentation should be
% available at \url{http://www.gsu.nd.edu} or at
% \url{http://graduateschool.nd.edu}.
%
% \subsection{Disclaimer}
% \label{sec:disclaimer}
% It could be thought of as suspicious if I begin with a disclaimer, but it is
% important for you to keep in mind that only
% {\em You} are responsible for the correct formatting of the document
% even though use of this class simplifies this task considerably. There
% are certain formatting things which need to be done manually and are
% described later in section \ref{sec:author}.
% As such, this class and its associated
% documentation {\em must not} be assumed to be a replacement of the
% formatting guide from the Graduate School and the official guide must be
% consulted, in case of doubt.
%
% In short, no one but you (the user) accepts any responsibility for works
% that do not get approved by the Graduate School. Use of the
% \nddiss\space class file implicitly states acceptance of this policy. Having
% said that, a document produced by using this class (as described in the
% following sections) has a pretty good likelyhood
% of being accepted as it is.
%
% \subsection{Dependencies and Limitations}
% \label{sec:deps}
% This classfile depends on many other packages to be present in either
% the |TEXMF| tree (system or local) or the \LaTeX\space search path (defined by shell variable
% \verb+$TEXINPUTS+). A list of the essential packages is mentioned in section \ref{sec:features}.
%
% Although I have tested it with \LaTeX\space
% [2001/06/01], it should be backwards compatible with \LaTeX\space
% [1995/12/01] and higher
% as well. It is not possible for me to list the version of each
% package used within the class file and you might get errors if the package in
% your |TEXMF| tree is outdated.
%
% The classfile is limited in the sense that it will produce an acceptable
% document with the packages that I have tried and included by default. There
% are numerous packages you may want to use for your work, but they may have
% to be modified accordingly. Things lacking include support
% for {\sffamily subfigure} package and proper formatting of the captions in
% such an environment. Formatting of the captions could be much easier with the
% new {\sffamily caption}\footnote{\textsf{caption} package by Axel
% Sommerfeldt v3.0b[2004/05/16] and higher}
% in general, and is a thing-to-do for future versions.
%
%
% \section{Usage}
%
% The \nddiss\space document class can be used only with \LaTeXe\space native mode or
% later, by typing |\documentclass|\oarg{options}|{nddiss2e}| at the
% beginning of your \LaTeX\space source file. The available options for
% the use of the class are discussed in section \ref{subsec:options} below.
% These have been limited to a small number in order to obtain documents
% with similar formatting under \LaTeXe\space using this class, although the
% \nddiss\space class is based on the \textsf{book} class,
% which has many other options.
%
% \subsection{Options}
% \label{subsec:options}
%
% By default, all documents produced using this class
% are formatted in |letterpaper| size and |onesided|,
% |doublespaced| mode, as per
% the requirements of the Graduate School. If you wish to override these
% restrictions, appropriate changes to the class file
% would be needed.
%
% The most important of the options is |draft|, |review| or
% |final|. Exactly one of
% these {\em must} be used, otherwise you would surely get errors.
%
% \DescribeMacro{draft}
% Using |draft| option will enable the {\it draft} mode of the
% |book| class, thus making the processing of the document
% faster. As a result of this, the most visible change is that instead of the
% included figure, only its placement box is displayed. An appropriate header
% is included to indicate that the prepared document is a draft document. The
% purpose of the |draft| option is to obtain a fast and preliminary
% document showing the labels for citations, tables, figures etc. and a black
% solid rule highlighting the horizontal overflows. Such a document would be
% the one you would prepare for revising your text during writing stages.
%
% \DescribeMacro{review}
% The |review| option makes it possible to prepare a document that is
% one step closer to the final version. Almost all the formatting of final
% version is present, along with the labels and keys as in the |draft|
% option. A document prepared with |review| option would be the one to
% check for proper formatting and giving to your advisor if (s)he wished to
% suggest corrections.
%
% \DescribeMacro{final}
% The |final| style option will produce the
% document for the production of archival
% copies of the dissertation for submission to the Graduate School.
%
% \DescribeMacro{twoadvisors}
% If you have more than one advisor for your project/research, selecting the
% |twoadvisors| option
% would produce an appropriately formatted titlepage. The |\secondadvisor|
% macro command is used to specify the name of the second advisor.
%
% \DescribeMacro{numrefs}
% \DescribeMacro{textrefs}
% Exactly one of these options -- |numrefs| or |textrefs|,
% needs to be specified. |numrefs| results in a
% numbered citation sytle with {\sffamily natbib} and ``nddiss2e'' citation style
% file\footnote{|nddiss2e.bst| is a slight modificiation of |abbrvnat.bst|
% in the {\sffamily natbib} package}.
% Using |textrefs| changes the citation style to be similar to
% ``author-date'' style with the same files. If none of these options is
% specified, the default style of numbered citations (ie.\/ same as if
% |numrefs| was used) is used.
%
% Since the same set of package and
% style file results in differing citation format, it is {\em strongly}
% suggested to
% refer to the documentation |natnotes.dvi| in your |TEXMF| tree,
% to be aware of the
% various ways in which you can make a citation in your text.
%
% \DescribeMacro{10pt}
% \DescribeMacro{11pt}
% \DescribeMacro{12pt}
% The choice of the fontsize is only applicable with the |draft| option
% used. By default, the document will be prepared in the 10pt size for the
% |draft|
% style option. For the |review| and the |final| style options,
% the document is prepared in the 12pt fontsize and choosing any other fontsize
% option will be ignored.
%
% \DescribeMacro{noinfo}
% Using the |noinfo| option would disable the information page produced
% when the |review| or |final| style options are used. It is
% recommended that you do not disable the ``info'' page unless it is the final
% most copy/copies for submission to the Graduate School.
%
% Thus, |\documentclass[draft,12pt]{nddiss2e}| would produce a document in
% \textsf{draft} format in 12pt font size and \\
% |\documentclass[final,twoadvisors]{nddiss2e}|
% would produce a document (with modified title page) for final submission
% to the Graduate School.
%
% \section{Features}
% \label{sec:features}
% A number of packages are required by default and must be present in your
% \TeX\space search path. As far as possible, these have been tested for proper
% formatting style with the \nddiss\/ class file. The list includes
% \textsf{ifthen},
% \textsf{exscale},
% \textsf{xspace},
% \textsf{longtable},
% \textsf{indentfirst},
% \textsf{tabularx},
% \textsf{showkeys},
% \textsf{enumerate},
% \textsf{latexsym}.
% \textsf{epsfig},
% \textsf{color},
% \textsf{graphicx},
% \textsf{url},
% \textsf{setspace}\footnote{v6.7[2000/12/01] or above},
% \textsf{amsmath},
% \textsf{amssymb},
% \textsf{float},
% \textsf{lscape},
% \textsf{rotating},
% \textsf{booktabs} and
% \textsf{natbib}.
% I urge you to read the documentation of these packages available in
% the |TEXMF| tree, if you think you might use their features
% or want to tweak some advanced options.
%
% Other packages may or may not be appropriate for use with the
% \nddiss\space class when producing copies to be submitted to the
% Graduate School. Please be careful when using packages that change the
% default fonts, or the layout(s).
%
% In general, the official guidelines of the Graduate School are
% followed to as much extent as possible. This includes proper
% formatting of the title page and the abstract page (from the
% \textsf{ndthesis} package), numbering of the pages in the {\it
% frontmatter}, generation of properly formatted table of contents, list
% of figures etc., as well as bibliography at the end. As per the guide,
% number of different fonts and font sizes used in the thesis is kept to
% a minimum. The contents, all lists and the bibliography are
% single-spaced but the inter-line spacing for the rest of the document
% is double.
%
% \subsection{Generating PDF document}
% \label{subsec:pdfdoc}
%
% The \nddiss\space class also allows production of pdf documents with
% pdf\LaTeX\space.  In this case, the \textsf{hyperref} and
% \textsf{hypernat} packages are also required. These packages ensure
% that the generated pdf document contains internal as well as external
% links for citations and bookmarks. A document produced by this method
% also contains embedded fonts ({\it press quality} pdf) and is suitable
% for electronic submission to the library and for microfilm archiving.
% Although the most appropriate options for the \textsf{hyperref} are
% passed on, for advanced features refer to its documentation.
%
% For creating documents with figures, it is imperative that these are
% present in a pdf-like format (eg.\/ pdf or metapost) rather than
% the usual encapsulated postscript (eps)
% format. An easy way to convert your {\it eps} files to {\it pdf} files
% is to use the utility |epstopdf| or |eps2pdf|, which should be
% available on your unix-like distribution already.
%
% \section{Arrangement of contents}
%
% A dissertation or a thesis document must contain the following parts,
% in the order listed. Only those explicitly marked as optional may be
% omitted. Again, I must point out that the official guide must be
% referred and its guidelines override the order listed here.
%
% \begin{enumerate}
% \item Title Page
% \item Copyright page
% \item Abstract {\em (optional for Master's thesis)}
% \item Dedication {\em(optional)}
% \item Table of Contents
% \item List of Figures
% \item List of Tables
% \item List Symbols {\em (optional)}
% \item Preface {\em (optional)}
% \item Acknowledgments {\em (optional)}
% \item Text
% \item Appendix (or Appendices) {\em (optional)}
% \item Bibliography (or References, or Works cited)
% \end{enumerate}
%
% The macros and environments described below ease the formatting of these
% parts.
%
% \subsection{Title page}
%
% The title page is generated by the standard
% \LaTeX\space macro |\maketitle| with no arguments.
% This macro has been modified for providing a title page format
% required for dissertations/theses.
%
% Prior to invoking it in
% your document, you should declare --
% \begin{itemize}
% \item \DescribeMacro{\title\{\}} the title of the document using the
% |\title| macro (note: title must be in ALL caps, eg.
% |\title{THIS IS \||\ A TITLE IN TWO LINES}|, and you may use
% linebreaks within the title),
% \item \DescribeMacro{\author\{\}} your name (full and exactly as
% registered with the Graduate School) with the
% |\author| macro, (eg. |\author{Gary Graham Gordon-Graeme}|),
% \item \DescribeMacro{\work\{\}} whether the document is a {\em Thesis} or a {\em
% Dissertation} as
% argument of |\work| macro, (eg. |\work{Dissertation}|),
% \item \DescribeMacro{\degaward\{\}} the degree you're aiming for (in
% full) with the |\degaward| macro, (eg. |\degaward{Doctor of Philosophy}| or
% |\degaward{Master of Science\\in\\Engineering}|),
% \item \DescribeMacro{\degprior\{\}} all your prior degrees to go with your name
% with |\degprior| macro, (eg. |\degprior{B. S., M. S.}|),
% \item \DescribeMacro{\advisor\{\}} the name of your advisor as argument to
% |\advisor| macro,
% \item \DescribeMacro{\secondadvisor\{\}} the name of second advisor, if any, with
% |\secondadvisor| macro \footnote{this macro is necessary when,
% \textsf{twoadvisors} option is used while invoking \nddiss\space class},
% \item \DescribeMacro{\department\{\}} the name of the department in the
% argument of the |\department| macro,
% (eg. |\department{Gnulogical Engineering}|) and,
% \item \DescribeMacro{\degdate\{\}} the month and year of the defense of the
% thesis with the |\degdate| (eg. |\degdate{June 2004}|). If you forget to
% declare this, the current month/year combination will be automatically used.
% \end{itemize}
%
% \DescribeMacro{\maketitle}
% After defining the above macro arguments, use |\maketitle| to generate
% a title page, which includes your entries.  All the above macros are
% required and if missing, they may result in errors in the generation
% of the title page.
%
% \subsection{Copyright page}
% \DescribeMacro{\makecopyright}
% \DescribeMacro{\makepublicdomain}
% The |\makecopyright| macro should be invoked after |\maketitle| to
% produce a copyright page. Alternatively, you can use
% |\makepublicdomain| to produce a page with the message ``This document
% is in the public domain.'' Note that the absence of the copyright page
% does {\em not} place your dissertaion in the public domain, you must
% declare it as such explicitely.
%
% \DescribeMacro{\copyrightholder\{\}}
% \DescribeMacro{\copyrightyear\{\}}
% Prior to calling |\makecopyright|, you may specify a different name
% for the copyright holder (the default is the name given through the
% |\author| macro) and for the copyright year (the default being the
% current year). You should do this with |\copyrightholder|\marg{name}
% and |\copyrightyear|\marg{year} macros.
%
% \subsection{Abstract page(s)} \DescribeEnv{abstract}
% The \textsf{abstract} environment has been modified from the default
% in the \textsf{report} class to comply with the requirements
% of the Graduate School. The abstract text should be placed between
% |\begin{abstract}| and |\end{abstract}|.
% In this environment, the author's name is placed in
% the top-right header, if it exceeds one page.
%
% \DescribeMacro{\abstractname\{\}}
% You may use |\abstractname|\marg{text} to change the abstract caption
% to |text|.
% Default name: |Abstract|.
%
% \subsection{Dedication}
%
% \DescribeEnv{dedication}
% The format of dedication is essentially free, but you may want to use
% the \textsf{dedication} environment for this purpose. This environment
% will center the text of your dedication vertically on the page. The
% dedication is optional.
% \DescribeMacro{\dedicationame\{\}}
% |\dedicationname|\marg{text} may be used to change the title for the
% dedication page. Default name: |\mbox{}| ie.\/ an empty title.
%
% \subsection{Table of contents, figures and tables}
% \DescribeMacro{\tableofcontents}
% \DescribeMacro{\listoffigures}
% \DescribeMacro{\listoftables}
% Use the macros |\tableofcontents|,|\listoffigures| and |\listoftables|
% in this order, to produce the required table of contents and list of
% figures and tables. (Note: the ``list of figures'' should precede the
% ``list of tables'' as per the Graduate School guide)
%
% \DescribeMacro{\contentsname\{\}}
% \DescribeMacro{\listfigurename\{\}}
% \DescribeMacro{\listtablename\{\}}
% The macros |\contentsname|, |\listfigurename| and |\listtablename| may
% be used to change the caption for Table of Contents, List of Figures
% and List of Tables, respectively. By default, they are named as |CONTENTS|,
% |FIGURES| and |TABLES|.
%
% \subsection{List of symbols}
% \DescribeEnv{symbols} The \textsf{symbols} environment is useful in
% formatting the list of symbols/abbreviations used in your work. It
% takes optional argument specifying the desired format, eg.
% |\begin{symbols}[cl]| for first column centered and the next column
% aligned left.  \DescribeMacro{\symbolsname\{\}} As earlier, the
% caption for the list of symbols can be changed by using the
% |\symbolsname| macro. Default name: |SYMBOLS|
%
% \DescribeMacro{\sym\{\}\{\}}
% Another macro |\sym|\marg{symbol}\marg{definition} makes the task of
% entering the symbols and their meanings in the \textsf{symbols}
% environment easier. |\sym| takes two arguments - first, a math
% ``object'' and second, assumed to be the plain text describing the
% symbol. Any plain text in the first argument needs to be set with
% |\mathrm{..}| and any math symbol in the second needs to placed in
% |$..$|.  Example: |\sym{\beta_\mathrm{norm}}{Definition for $\beta$}|
%
% \subsection{Preface}
% \DescribeEnv{preface}
% \DescribeMacro{\prefacename\{\}}
% An environment \textsf{preface} is provided for formatting the preface to the
% document.  The name of this {\em chapter} may be changed by |\prefacename|
% macro. Default name: |PREFACE|
%
% \subsection{Acknowledgments}
% \DescribeEnv{acknowledgments}
% \DescribeMacro{\acknowledgename\{\}}
% The environment \textsf{acknowledgments} is used to format the
% acknowledgment {\em chapter}. As above |\acknowledgename| macro may be used
% to change the title name. Default name: |ACKNOWLEDGMENTS|.
%
% \subsection{Text} \DescribeMacro{\mainmatter}
% Use the macro |\mainmatter| at the beginning of the text sections ie.
% all text matter should follow this macro as in the {\sffamily book}
% class. The text is formatted in |\normalspacing| ie. double-spacing.
% The pages are numbered in |plain| pagestyle such that the page numbers
% are centered in the bottom. The |chapter| titles can be multi-line or
% long and would be formatted doubly spaced.
%
% \subsection{Appendix}
% \DescribeMacro{\appendix}
% As in the {\sffamily book} class, use the |\appendix| to mark the
% end of the last chapter in the main section and the start of the
% appendices. To begin an appendix, using a |\chapter|\marg{title}
% macro. Appendices will be automatically ``numbered'' alphabetically.
%
% \subsection{Backmatter}
% \DescribeMacro{\backmatter}
% This macro separates the bibliography, index
% and glossary from the main matter and appendices, if any.
%
% \subsection{Bibliography} \DescribeMacro{\bibliography}
% You may use the |\bibliography|\marg{bibfile} macro to generate the
% bibliography with \BibTeX\/. In order to use the
% \BibTeX\/ path for generating bibliography, one needs to have
% all the bibliographic data in \BibTeX\/  files. You should
% refer to \BibTeX\/ manual for details about making a |.bib| file
% and format for the entries.
%
% \DescribeMacro{\bibname\{\}} The default name for this {\em chapter} is
% |BIBLIOGRAPHY|. You may change it by using the macro command
% |\bibname|\marg{newbibname}.
%
% \DescribeEnv{thebibliography}
% Alternatively, you can also make your own bibliography by using
% {\sffamily thebibliography} environment. In this case, you would have
% to write the reference entries in the right format in your |.tex| source
% file itself.
%
% For citing references in the text, the package {\sffamily natbib} is
% included with options: \texttt{numbers,sort\&compress} (|numrefs| option) or
% \texttt{authoryear,sort} (|textrefs| option). The package \textsf{natbib} is a
% fantastic package that has numerous macros for {\em citing} in different ways.
% It would make your life a lot easier if you read up its documentation. In
% addition, for generation of pdf document, another package {\sffamily
% hypernat} helps generates internal links for references.
%
% {\bfseries Warning:} The packages {\sffamily cite} and \textsf{citation} are
% NOT compatible
% with the {\sffamily natbib} package, and they must not be used.
%
% \section{Note for the authors}
% \label{sec:author}
% As noted earlier, the dissertation author must make sure that the
% following conditions are met in order to generate a dissertation
% acceptable by the Graduate School:
%
% \begin{itemize}
% \item List of Figures should be {\em before} List of Tables, ie. the
% macro command |\listoffigures| comes before |\listoftables| in the
% frontmatter.
% \item Chapter titles need to be written in ALL CAPS.
% \item Table caption must be {\em above} the table, but the figure
% caption should be {\em below} it.
% In case of {\sffamily table} environment, this can be achieved by
% putting |\caption| before you include the table (eg. in a {\sffamily
% tabular} environment). In the {\sffamily figure} environment, |\caption|
% goes after the |\includegraphics| macro command.
% \item Table captions need to be in ALL CAPS as well.
% \item Bibliography is the last section/chapter of the thesis.\footnote{Except
% if you are using {\em chapter-wise} bibliography}
% \end{itemize}
%
% \subsection{Chapter-wise bibliography}
% It is now possible to obtain bibliography as a section in each chapter
% (as is common
% in some departments esp.\/ humanities) with some minor code
% implementation in the |.tex| files. To do this you have to load the
% package {\sffamily chapterbib} (without any options)
% in the preamble of your main tex file and
% redefine some commands as below -
% \begin{verbatim}
% %% Main source file %%
% \documentclass[...]{nddiss2e}
% \usepackage{chapterbib}
% \renewcommand{\bibname}{Cited works}
% \renewcommand{\bibsection}{\section{\bibname}}
% ...
% \begin{document}
% \include{chptr1}
% ...
% \include{appndx}
% \end{document}
% \end{verbatim}
%
% In such a case, you must separate the chapters or sections in
% which you want individual bibliographies in different files and {\em include}
% them in the main file as above. Each such |\include|d file must contain
% its own |\bibliographystyle{nddiss2e}| and |\bibliography{...}| command at
% an appropriate position. There should not be any bibliographic commands in
% the main source file.
%
% After compiling the main tex file once (with |latex| or
% |pdflatex|)\footnote{This creates |.aux| files which are needed by |bibtex|},
% you
% would have to run |bibtex| on each of the separate files to obtain a |.bbl|
% for each file. The remaining steps are the same as for a normal |.tex| file.
%
% You can find more details of this in
% the {\sffamily natbib} manual (|natbib.dvi|) in the |TEXMF| tree.
% \subsection{Tips and suggestions}
% \label{subsec:tips}
% \begin{itemize}
% \item Use |\toprule|, |\midrule| and |\bottomrule| macro commands (from the
% {\sffamily booktabs} package) in the
% tables for generating the appropriate horizontal rules. Use of vertical rules
% to separate columns in tables, is in general, bad style and should be avoided as much as possible.
% \item Use the {\sffamily longtable} environment for handling very long
% tabular materials. Again, use the vertical rules only if very necessary.
% \begin{verbatim}
% \begin{longtable}{lc}
% \caption[]{LONG TABLE CAPTION \label{tab:longtable} }
% \toprule
% Heading 1 & Heading 2 \\
% \midrule
% \endfirsthead
% \caption[]{ {\em Continued} } \\
% \midrule
% Heading 1 & Heading 2 \\
% \midrule
% \endhead
% \endfoot
% \bottomrule
% \endlastfoot
% % Now the tabular material %
% Long & Table etc. \\
% \end{longtable}
% \end{verbatim}
% \item If a figure/table is very wide and will not fit normally,
% use the {\sffamily landscape} environment (from the included {\sffamily lscape}
% package) to format them in {\em landscape}
% mode. They will automatically appear on a separate page.
% The {\sffamily sidewaystable} environment (from the included {\sffamily
% rotating} package) is incompatible with the current
% class and must be avoided.
% \item Usually the width of the figure/table captions is 90\% of the textwidth
% (ie. |0.9\textwidth|),
% but if needed can be changed as per the following:
% \begin{itemize}
% \item Use a {\sffamily minipage} environment of appropriate width and enclose
% your tabular or figure float inside it, or
% \item set the |\capwidth| inside the {\sffamily table} or the
% {\sffamily figure} environment, and |\LTcapwidth| {\em outside} the {\sffamily
% longtable} environment, e.g.,\\[\baselineskip]
% \begin{minipage}{0.5\textwidth}
% \begin{verbatim}
% \begin{table}[H]
% \setlength{\capwidth}{0.8\textwidth}
% \centering
% \caption{TABLE CAP\label{tab:this}}
% \begin{tabular}{lc}
% ...
% \end{tabular}
% \end{table}
% \end{verbatim}
% \end{minipage}
% \begin{minipage}{0.5\textwidth}
% \begin{verbatim}
% \setlength{\LTcapwidth}{6in}
% \begin{longtable}{lccc}
% ...
% \end{\longtable}
% \end{verbatim}
% \end{minipage}
% \end{itemize}
% \item Use {\sffamily tabularx} environment for the actual
% formatting of the tables (within the \textsf{table} environment). It differ
% slightly from \textsf{tabular} environment and you should refer to their
% documentation in the |TEXMF| tree for more information.
% \item If you've used a {\sffamily longtable} environment in your document,
% it might be
% necessary to compile the document multiple times so as to get proper
% alignment of columns. This is documented in the {\sffamily longtable} manual.
% \item If you wish to use |\footnotes| in the \textsf{longtable} environment,
% please read its documentation. There are some handicaps present.
% \item A new macro command |\subsubsection| (self-explanatory) has been defined.
% It is numbered as |3.2.1.4| in the TOC.
% \item To cite a website in your bibliography\footnote{More info at
% \url{http://www.tex.ac.uk/cgi-bin/texfaq2html?label=citeURL}}, use the
% following format in your |.bib| file:
%  \begin{verbatim}
%      @Misc{fairley2000,
%      author =       "N. Fairley",
%      title =        "Casa{XPS} {VAMAS} processing software",
%      howpublished = "Website",
%      note =         "\url{http://www.casaxps.com}",
%      }
%  \end{verbatim} This when processed with the |nddiss2e.bst| citation style file
% gives:\newline
% 111. N. Fairley. CasaXPS VAMAS processing software. Website.
% \url{http://www.casaxps.com}.\newline
% \end{itemize}
%
% \subsection{You found errors?}
% If you find some errors in formatting of your document, most likely these are
% NOT due to the \nddiss\/ class, but due to either wrongly used
% commands/macros or conflicting commands/macros in a package that you might
% have used ie.\/ a mistake from your side. Please verify that before
% contacting anyone regarding the class file.
%
% Read the manuals for the packages that you might have used and find out if
% they have macro commands that modify the page-layout, spacing etc.\/ or if
% they conflict with already defined commands (eg.\/ packages
% \textsf{natbib} and
% \textsf{cite} redefine the same command |\cite| in different ways). I would
% suggest that you {\em read the manuals anyway}! It is a good practice.
%
% In the possible scenario that you
% have found a significant error, please try to find out the source of the
% error and, even better, a possible alternative, please
% report it to the Graduate School's reviewer(s). It would also be great, if
% you try to find a solution for the problem and inform that as well.
% Please also inform if you find this documentation confusing or misleading or
% if any mistakes are present.
%
% Your doing so will ensure that the reviwer(s) are aware of the problem and
% its possible solutions and they can give better answers to the people who
% might encounter similar problems. It will also help making the future
% versions of the class file better.
%
% \section{Example}
% Below is a basic |.tex| sample for your help.
% A file called |template.tex|\footnote{Most likely present in the same
% directory as |nddiss2e.cls| ie.\/ in |TEXMF/tex/latex/nddiss2e/|}
% is generated from |nddiss2e.ins| which might serve as a guide for
% your document. The |example| directory contains a mock thesis modified
% from {\sffamily ndthesis} classfile and should also be helpful.
%
% \begin{verbatim}
% \documentclass[numrefs,final]{nddiss2e}
%
% \begin{document}
%
% \frontmatter
%
% \title{}
% \author{}
% \work{Dissertation}
% \degprior{B.S., M.S.}
% \degaward{Doctor of Philosophy}
% \advisor{}
% \department{}
% \maketitle
% \makepublicdomain
%
% \begin{abstract}
% Abstract here
% \end{abstract}
% \renewcommand{\dedicationname}{\mbox{}}% Empty dedication title
% \begin{dedication}
% For Someone
% \end{dedication}
%
% \tableofcontents
% \listoffigures
% \listoftables
%
% \begin{preface}
% Preface here
% \end{preface}
% \begin{acknowledge}
% Thanks to everyone
% \end{acknowledge}
% \begin{symbols}
% \sym{a}{definition of a}
% \end{symbol}
%
% \mainmatter
%
% \chapter{One} % Chapter 1
% All the text ...
%
% \appendix
% \chapter{Additional data} % Appendix A
%
% \backmatter
% \bibliographystyle{nddiss2e}
% \bibliography{bibdatabase}
%
% \end{document}
% \end{verbatim}
%
%
%
% \StopEventually{%
%   \footnotesize
%   \PrintChanges
%   }
%
% \clearpage
% \section{The Implementation}
% Following is my attempt at documenting the source of the
% \nddiss\space class file for the \TeX\space hackers.
%
% \small\mbox{}\\
% \iffalse (Don't show this in the doc).
%<*class>
% \fi
% At start, we define the base version of \LaTeXe\space needed and the label
% information for the \nddiss\space class.
%    \begin{macrocode}
\NeedsTeXFormat{LaTeX2e}[1999/12/01]
\ProvidesClass{nddiss2e}
%    \end{macrocode}
% \iffalse
%</class>
% \fi
% \iffalse (Don't show this in the doc).
%    \begin{macrocode}
%<template>\ProvidesFile{template.tex}
%    \end{macrocode}
% \fi
%    \begin{macrocode}
    [2005/07/27 v3.0 ^^J%
%    \end{macrocode}
% \iffalse
%<*class>
% \fi
%    \begin{macrocode}
     Notre Dame Dissertation document class by Sameer Vijay^^J]
%
%    \end{macrocode}
% \begin{macro}{\dissfileversion}
% \begin{macro}{\dissfiledate}
% The |\dissfileversion| and |\dissfiledate| macros contain the version and the date
% of the release.
%    \begin{macrocode}
\providecommand{\dissfileversion}{3.0}
\providecommand{\dissfiledate}{2005/07/27}
%
%    \end{macrocode}
% \end{macro}
% \end{macro}
% \iffalse
%</class>
% \fi
% \iffalse (Don't show this in the doc).
%    \begin{macrocode}
%<template>     Template file for NDdiss2e class by Sameer Vijay^^J]
%    \end{macrocode}
% \fi
% \iffalse
%<*class>
% \fi
% New boolean variables for the options used in \nddiss\ class are
% set here with default values.
%    \begin{macrocode}
\newif\ifdiss@draft             \diss@drafttrue
\newif\ifdiss@review            \diss@reviewfalse
\newif\ifdiss@final             \diss@finalfalse
\newif\ifinfo@page              \info@pagetrue
\newif\ifadvisors@two           \advisors@twofalse
\newif\ifdiss@dedication        \diss@dedicationfalse
\newif\ifnum@refs               \num@refstrue
\newif\ifcentered@chaptitle     \centered@chaptitletrue
%
%    \end{macrocode}
% \begin{macro}{draft}
% \begin{macro}{review}
% \begin{macro}{final}
% Exactly one of these options must be present in order to get a proper document. These
% options set appropriate boolean variables (flags) and pass some common
% options to the parent {\sffamily book} class.
%    \begin{macrocode}
\DeclareOption{draft}{
    \setlength\overfullrule{5pt}
    \typeout{DRAFT MODE}\typeout{}\info@pagefalse%
    \diss@drafttrue\diss@reviewfalse\diss@finalfalse
    \PassOptionsToClass{letterpaper,oneside,draft}{book} }
%
\DeclareOption{review}{
    \typeout{REVIEW MODE}\typeout{}\info@pagetrue%
    \diss@draftfalse\diss@reviewtrue\diss@finalfalse
    \PassOptionsToClass{12pt,letterpaper,oneside,final}{book} }
%
\DeclareOption{final}{
    \setlength\overfullrule{0pt}
    \typeout{FINAL MODE}\typeout{}\info@pagetrue%
    \diss@draftfalse\diss@reviewfalse\diss@finaltrue
    \PassOptionsToClass{12pt,letterpaper,oneside,final}{book} }
%
%    \end{macrocode}
% \end{macro}
% \end{macro}
% \end{macro}
% \begin{macro}{numrefs}
% \begin{macro}{textrefs}
% The options |numrefs| or |textrefs| selects appropriate citation style ie.\/
% ``numbered'' or ``textual'', respectively. By
% choosing |textrefs|, one can get ``author-date'' style of citation in the
% text. The default is |numrefs|.
%    \begin{macrocode}
\DeclareOption{numrefs}{
  \typeout{NUMBERED REFERENCES}\num@refstrue}
\DeclareOption{textrefs}{
  \typeout{TEXTUAL REFERENCES}\num@refsfalse}
%    \end{macrocode}
% \end{macro}
% \end{macro}
% The option |nocenter| (not documented) allows non-centered chapter titles.
%    \begin{macrocode}
\DeclareOption{nocenter}{\centered@chaptitlefalse}
%
%    \end{macrocode}
% The |openbib| option (not documented) is useful
% in creating indented bibliography.
% Usually you would not need to use this option since the default layout of the
% |bibliography| is very much acceptable.
%    \begin{macrocode}
\DeclareOption{openbib}{%
   \PassOptionsToPackage{openbib}{natbib}
}
%
%    \end{macrocode}
% The other options are declared in the following lines.
% \begin{macro}{twoadvisors}
% The |twoadvisors|
% option sets the flag for modifying the layout of the title page.
%    \begin{macrocode}
\DeclareOption{twoadvisors}{\typeout{TWO ADVISORS}\typeout{}%
    \advisors@twotrue}
%
%    \end{macrocode}
% \end{macro}
% \begin{macro}{10pt}
% \begin{macro}{11pt}
% \begin{macro}{12pt}
% The options |10pt|, |11pt| or |12pt| are appropriately passed on to the
% {\sffamily book} class depending on whether |\diss@draft| flag is set true.
%    \begin{macrocode}
\DeclareOption{10pt}{%
 \ifdiss@draft%
  \PassOptionsToClass{10pt}{book}%
 \else%
  \OptionNotUsed%
  \ClassWarningNoLine{nddiss2e}%
    {Font size 10pt not allowed; using 12pt}%
 \fi%
}
\DeclareOption{11pt}{%
 \ifdiss@draft%
  \PassOptionsToClass{11pt}{book}%
 \else%
  \OptionNotUsed%
  \ClassWarningNoLine{nddiss2e}%
    {Font size 11pt not allowed; using 12pt}%
 \fi
}
\DeclareOption{12pt}{%
   \PassOptionsToClass{12pt}{book}%
}
%
%    \end{macrocode}
% \end{macro}
% \end{macro}
% \end{macro}
%    \begin{macrocode}
\DeclareOption{noinfo}{\info@pagefalse}
%
%    \end{macrocode}
% The |twoside| option (not documented) is when you want to prepare a two-sided
% document for your own use. The only difference from the one-sided document
% is in the page layout. This option is passed on to the parent {\sffamily
% book} class.
%    \begin{macrocode}
\DeclareOption{twoside}{\typeout{TWO SIDED DOCUMENT}%
  \PassOptionsToClass{twoside}{book} }%
%
%    \end{macrocode}
% All options other than those defined above are ignored and
% a warning is printed
% on the screen during compile-time. After processing all the options, the
% {\sffamily book} class is loaded with the specified options.
%    \begin{macrocode}
\DeclareOption*{\ClassWarning{nddiss2e}%
    {UnknownOption `\CurrentOption'} }%
\ProcessOptions\relax
\LoadClass{book}
%
%    \end{macrocode}
% At this stage, the packages {\sffamily ifthen}, {\sffamily exscale},
% {\sffamily ifpdf}, {\sffamily longtable}, {\sffamily xspace}, {\sffamily
% indentfirst}, {\sffamily tabularx}, {\sffamily enumerate} and {\sffamily
% latexsym} are loaded. It is important to load these in specific order so as
% not to cause conflicts in definitions of certain macros.
%    \begin{macrocode}
\RequirePackage{ifthen,exscale}
\RequirePackage{ifpdf}
\RequirePackage{longtable}
\RequirePackage{xspace}
\RequirePackage{indentfirst}
\RequirePackage{tabularx}
\RequirePackage{enumerate}
\RequirePackage{latexsym}
%
%    \end{macrocode}
% If the |\diss@final| is set false (when using |draft| or |review| option) then
% {\sffamily showkeys} package is also loaded.
%    \begin{macrocode}
\ifdiss@final\relax\else\RequirePackage{showkeys}\fi
%
%    \end{macrocode}
% Depending in whether you are using pdf\LaTeX\space or plain \LaTeX\space,
% {\sffamily epsfig}, {\sffamily color} and {\sffamily graphicx} are loaded
% with respective options. Additionaly when using pdf\LaTeX\space, the package
% {\sffamily hyperref} (for internal/external links in the document)
% is also loaded. The options for this package have been tested to produce
% a document which can be printed on laser printers without any problems
% because of colored link boxes.
%    \begin{macrocode}
\ifpdf
  \RequirePackage[pdftex]{epsfig}
  \RequirePackage[pdftex]{color}
  \RequirePackage[pdftex]{graphicx}
  \RequirePackage[pdftex,
            letterpaper=true,%
            bookmarks=true,%
            bookmarksnumbered=true,%
            linktocpage=true,%
            breaklinks=true,%
            bookmarkstype=toc,%
            colorlinks=false,%
            pdfpagemode=UseOutlines]{hyperref}
  \AtBeginDocument{
  \pdfadjustspacing=1
  }
\else
  \RequirePackage[dvips]{epsfig}
  \RequirePackage[dvips]{color}
  \RequirePackage[dvips]{graphicx}
\fi
%
%    \end{macrocode}
% Now the {\sffamily natbib} package is loaded
% with its options, appropriate to |numrefs| or |textrefs| class option. If |numrefs| is
% specified, then {\sffamily natbib} is read-in with its options for
% ``numbered'' references and sorted \& compressed (eg. |[3-6,8-10]|).
% In this case, the default delimiter is square brackets and the
% default seperator is a comma. For the |textrefs| option, the
% {\sffamily natbib} package is read-in so as to sort the references in an
% ``author-date'' style of citations.
% The default delimitor and separator, in this case, are round brackets and
% colon, respectively.
%
% For creating internal links in a pdf document with
% pdf\LaTeX\space, the package {\sffamily hypernat} is also loaded.
%    \begin{macrocode}
\ifnum@refs
  \RequirePackage[numbers,sort&compress]{natbib}
\else
  \RequirePackage[authoryear,sort]{natbib}
\fi
%
\ifpdf
    \RequirePackage{hypernat}
\fi
%
%    \end{macrocode}
% Additionaly, the packages {\sffamily amsmath}, {\sffamily amssymb},
% {\sffamily float}, {\sffamily lscape}, {\sffamily booktabs}, {\sffamily
% rotating}, {\sffamily url} and {\sffamily setspace} are loaded when (pdf)\LaTeX\space
% processes |\begin{document}|. Again, the order of these packages is
% important.
%    \begin{macrocode}
\AtBeginDocument{
\RequirePackage{amsmath,amssymb}
\RequirePackage{float}
\RequirePackage{lscape}
\RequirePackage{booktabs}
\RequirePackage{rotating}
\RequirePackage{url}
\RequirePackage[doublespacing]{setspace}[2000/12/01]
}
%
%    \end{macrocode}
% Set the |\pagestyle| for the document to |plain| here
% and define default spacing.
%    \begin{macrocode}
\AtBeginDocument{
\pagestyle{plain}
\normalspacing
\typeout{Pagestyle and spacing normal}
}
%
%    \end{macrocode}
% Here, define some spacing macros for page layout and doublespacing.
%    \begin{macrocode}
\newcommand{\normalspacing}{\doublespacing}
\newcommand\single@baselinestretch{1.0}
\newcommand\double@baselinestretch{1.66}
\newlength{\single@skip}
\setlength{\single@skip}{\single@baselinestretch em}
\newlength{\double@skip}
\setlength{\double@skip}{\double@baselinestretch em}
%
%    \end{macrocode}
% Define new lengths for some variables for a proper layout of normal pages,
% pages with text and figures and pages with only floats.
%    \begin{macrocode}
\setlength{\hoffset}{0pt}
\setlength{\voffset}{0pt}
\addtolength{\topmargin}{-32pt}
\setlength{\headsep}{12pt}
\setlength{\marginparwidth}{54pt}
\setlength{\marginparsep}{0pt}
\addtolength{\textheight}{63pt}
\addtolength{\textwidth}{26pt}
\setlength{\oddsidemargin}{36pt}
\setlength{\evensidemargin}{18pt}
\setlength{\footskip}{36pt}
%
\setlength{\floatsep}{30pt}
\setlength{\intextsep}{50pt}
%
%    \end{macrocode}
%    \begin{macrocode}
\newcommand{\clearemptydoublepage}{\newpage{\pagestyle{empty}%
    \cleardoublepage}}
%
%    \end{macrocode}
% \begin{macro}{\nddiss} Define the macro |\nddiss| that is the logo
% used in the titlepage and the stamp in the dissertation document.
%    \begin{macrocode}
\DeclareRobustCommand{\nddiss}{%
        \textsf{{\scshape nd}diss}\kern-0.03em%
        2$_\mathsf{\textstyle\varepsilon}$}
%
%    \end{macrocode}
% \begin{macro}{\work}
% \begin{macro}{\degaward}
% \begin{macro}{\degprior}
% \begin{macro}{\advisor}
% \begin{macro}{\secondadvisor}
% \begin{macro}{\department}
% \begin{macro}{\degdate}
% Here define new macros for use in the dissertation title page.
%    \begin{macrocode}
\newcommand{\work}[1]{\def\@work{#1}}
\newcommand{\degaward}[1]{\def\@degaward{#1}}
\newcommand{\degprior}[1]{\def\@degprior{#1}}
\newcommand{\advisor}[1]{\def\@advisor{#1}}
\ifadvisors@two
    \newcommand{\secondadvisor}[1]{\def\@secondadvisor{#1}}
\fi
\newcommand{\department}[1]{\def\@department{#1}}
\newcommand{\degdate}[1]{\def\@degdate{#1}}
  \degdate{\ifcase\month\or
    January\or February\or March\or April\or May\or June\or
    July\or August\or September\or October\or November\or December\fi
    \space\number\year}
%
%    \end{macrocode}
% \end{macro}
% \end{macro}
% \end{macro}
% \end{macro}
% \end{macro}
% \end{macro}
% \end{macro}
% \end{macro}
% As a default, these macros have an empty arguement. Only the |\degdate| macro
% takes on the current month-year combination in the absence of any assignation.
%    \begin{macrocode}
% Defaults are empty except the \degdate
  \title{}
  \author{}
  \work{}
  \degaward{}
  \degprior{}
  \advisor{}
  \ifadvisors@two \secondadvisor{} \fi
  \department{}
%
%    \end{macrocode}
% \begin{macro}{\@infopage}
% Define |\@infopage| macro that will create a page which contains
% important information about the document and the version of \nddiss\space
% used etc.\ for the end-user and the proofreader along with a standard
% disclaimer and
% details of where to find documentation for the \nddiss\space class
% file. This information can be suppressed by specifying ``|noinfo|'' option
% while invoking the \nddiss\space class.
%    \begin{macrocode}
\DeclareRobustCommand{\@infopage}{
  \thispagestyle{empty}
  \null\vspace*{\single@skip}
  \begin{center}
  \large This \@work\space \\ entitled \\ \@title \\
        typeset with \nddiss\ v%
      \dissfileversion\ (\dissfiledate) %
      on \today\space for\\
  \@author\\
  \end{center}

   \normalfont\normalsize\singlespacing

   \noindent This \LaTeXe\space classfile conforms to the
   University of Notre Dame style guidelines established in
   Spring  2004. However it is still possible to generate a
   non-conformant document if the instructions in the class
   file documentation are not followed!

   \begin{center}
   \begin{minipage}{0.75\textwidth}
   \noindent Be sure to refer to the published Graduate
   School guidelines at \url{http://graduateschool.nd.edu}
   as well. Those guidelines override everything mentioned
   about formatting in the documentation for
   this \nddiss\space class file.
   \end{minipage}
   \end{center}

   \noindent It is YOUR responsibility to ensure that the Chapter titles
   and Table caption titles are put in CAPS LETTERS. This classfile does
   {\em NOT\space} do that! \\

  \noindent\itshape This page can be disabled by
  specifying the ``{\upshape\ttfamily noinfo}'' option to the class invocation.
  \upshape
(i.e.,{\ttfamily{\textbackslash}documentclass[\ldots,noinfo]\{nddiss2e\}}
)
  \begin{center}
    {\bfseries\large\singlespacing This page is \slshape NOT
    \upshape part of the dissertation/thesis, but MUST be turned in to the
    proofreader(s) or the reviwer(s)!}
  \end{center}
    \normalsize\normalfont
    \nddiss\ documentation can be found at these locations:
  \begin{center}
    \url{http://www.gsu.nd.edu}\\
    \url{http://graduateschool.nd.edu}
  \end{center}

\vfill
\normalfont\normalsize\normalspacing\eject}
%
%    \end{macrocode}
% \end{macro}
% \begin{macro}{\maketitle}
% Redefine the macro |\maketitle| to produce the information page as well as
% the actual title page of the dissertation.
%    \begin{macrocode}
\renewcommand{\maketitle}{
  \ifinfo@page\@infopage\else\relax\fi%
  \clearemptydoublepage
  \normalfont\normalsize\normalspacing
%    \end{macrocode}
% \end{macro}
% \begin{macro}{titlepage}
% The structuring begins with checking the proper macros for
% obtaining a correct formatting for the title page. If any of those are
% not defined, an error is issued and processing stopped. Most of the code for
% this was taken from the earlier {\sffamily ndthesis} class and hence, the
% documentation is also picked from there.
%    \begin{macrocode}
  \begin{titlepage}%
\ifthenelse{\equal{\@work}{}}{\ClassError{nddiss2e}%
  {The \protect\work\space macro is undefined.\MessageBreak
        The title page may be incorrectly formatted.}%
  {Specify \protect\work\space as Dissertation or Thesis}}{\relax}
\ifthenelse{\equal{\@degaward}{}}{\ClassError{nddiss2e}%
  {The \protect\degaward\space macro is undefined.\MessageBreak
        The title page may be incorrectly formatted.}%
  {Specify \protect\degaward\space. It defines the awarded degree%
         (Ph.D., M.S., etc.)}}{\relax}
\ifthenelse{\equal{\@advisor}{}}{\ClassError{nddiss2e}%
  {The \protect\advisor\space macro is undefined.\MessageBreak
        The title page may be incorrectly formatted.}%
  {Spepcify \protect\advisor\space It is who signs your walking papers!}}{\relax}
\ifthenelse{\equal{\@department}{}}{\ClassError{nddiss2e}%
  {The \protect\department\space macro is undefined.\MessageBreak
        The title page may be incorrectly formatted.}%
  {Specify which \protect\department\space is awarding your degree?}}{\relax}
\ifadvisors@two
  \ifthenelse{\equal{\@secondadvisor}{}}{\ClassError{nddiss2e}%
  {The \protect\secondadvisor\space macro is undefined.\MessageBreak
            The title page may be incorrectly formatted.}%
  {Use \protect\secondadvisor\space for your second advisor}}{\relax}
\fi
%
%    \end{macrocode}
% \end{macro}
% Now set up some skip registers to
% hold the inter-data spacing.  The initial values will vertically center
% the title page provided the title is only one line long.  We'll account for
% the longer ones in a bit \ldots
%    \begin{macrocode}
\skip1=1.2\double@skip
\skip2=1.7\double@skip
\skip3=2.7\double@skip
\skip4=1em
%
%    \end{macrocode}
% If the author has two advisors, we need to do a little tweaking.
%    \begin{macrocode}
\ifadvisors@two \skip1=1.0\double@skip \skip4=1.5em \else\relax\fi
%
%    \end{macrocode}
% For positioning the title, we first stuff it in a box and find out how
% many lines it will occupy.  In order to maintain the $1''$ top margin
% required by Notre Dame, we must not move our defaults up by more than
% one-half of a |\doubleskip|.  (Actually, there's a bit more room than
% that, but we reserve that space in the case of multiple advisors.)  As it
% turns out, if the title is just two lines long, then just moving the top
% of the text by |.5\doubleskip| will cause it to remain centered.  For
% longer titles, the extra lines have to be taken out of the inter-section
% spacing (skip register 1).
%    \begin{macrocode}
\setbox0=\hbox{\@title}
\ifdim \wd0 > 3\hsize
   \advance\skip4 .5\double@skip
   \advance\skip1 -.33\double@skip
\else
   \ifdim \wd0 > 2\hsize
     \advance\skip4 .5\double@skip
         \advance\skip1 -.167\double@skip
  \else
     \ifdim \wd0 > \hsize
        \advance\skip4 .5\double@skip
         \fi
  \fi
\fi
%
%    \end{macrocode}
% Our default assumes a two-line \textsf{degree} field such as
% \begin{center}
% Doctor of Philosophy \\ in Somekind of Science \\
% \end{center}
% but we check to see if it is simply one
% line long.  If so, we need to add-back that extra line to the spacing.
%    \begin{macrocode}
\setbox1=\vbox{\@degaward}
\ifdim \ht1 > \double@skip\relax
   \else \advance\skip1 .167\double@skip \fi
%
%    \end{macrocode}
% Finally we start putting the text in place \ldots centered, of course.
%    \begin{macrocode}
  \null\vspace*{-\skip4}
  \begin{center}%
    \@title \par%
    \vskip\skip1%
%
%    \end{macrocode}
% Now skip the required vertical space, declare that this is for the
% University of Notre Dame, and list what degree has been earned.
%    \begin{macrocode}
    A \@work \par%
    \vskip\skip1%
      Submitted to the Graduate School \\
          of the University of Notre Dame \\
          in Partial Fulfillment of the Requirements \\
          for the Degree of \par
        \vskip\skip1%
        \@degaward%
        \vskip\skip1%
        by%
        \vskip\skip1%
%
%    \end{macrocode}
% Now format the author's name and credentials.
%    \begin{macrocode}
    {\normalsize
      \begin{tabular}[t]{c@{}l}%
        \@author, &\ \@degprior
      \end{tabular}\par }%
%
%    \end{macrocode}
% Now skip the proper space and place the signature line for the advisor
% with his/her name typeset below it.  This is accomplished by essentially
% centering a box that is twice as long as the required length of the
% signature line and placing the line in only the right-hand side.
%    \begin{macrocode}
     \ifadvisors@two
        \vskip\skip2\else
        \vskip\skip3\fi%
        \hspace*{2.75in}\underline{\hspace{2.75in}}\\%
        \hspace*{2.75in}\@advisor, Director\\
%
%    \end{macrocode}
% If there is a second advisor, place that line here now.
%    \begin{macrocode}
\ifadvisors@two %
      \vskip\double@skip%
        \hspace*{2.75in}\underline{\hspace{2.75in}}\\%
        \hspace*{2.75in}\@secondadvisor, Director\\
\fi
%
%    \end{macrocode}
% Finally, we do a vertical fill to get the department and date to the page
% bottom.
%    \begin{macrocode}
  \vskip\skip1%
  Graduate Program in \@department \\%
  Notre Dame, Indiana \\
  \@degdate
  \end{center}
  \end{titlepage}%
}
%
%    \end{macrocode}
% \begin{environment}{copyrightpage} The environment |copyrightpage| defines
% the defaults for proper formatting the copyright page (if opted).
%    \begin{macrocode}
\newenvironment{copyrightpage}{%
  \clearemptydoublepage
  \typeout{Copyleft page}
  \pagestyle{empty}
  \null\vfil
  \begin{center}\normalspacing}%
{ \end{center}\vfil\null \clearpage }
%
%    \end{macrocode}
% \end{environment}
% \begin{macro}{\copyrightholder}
% \begin{macro}{\copyrightyear}
% Define a few macros for defining the copyright holder and the year
% desired. By default, they are taken as the current year and the |author|
% of the dissertation.
%    \begin{macrocode}
\newcommand{\@copyrightyear}{\year}
\newcommand{\@copyrightholder}{\@author}
\newcommand{\copyrightyear}[1]{\renewcommand{\@copyrightyear}{#1}}
\newcommand{\copyrightholder}[1]{\renewcommand{\@copyrightholder}{#1}}
%
%    \end{macrocode}
% \end{macro}
% \end{macro}
% \begin{macro}{\makecopyright} Finally, the |\makecopyright| macro creates the
% copyright page as per defined in the |copyrightpage| environment.
%    \begin{macrocode}
\newcommand{\makecopyright}{%
  \ifdiss@final
    \begin{copyrightpage}
    \normalfont\normalsize
    \copyright\space Copyright by \\
    \@copyrightholder \\
    \@copyrightyear\\
    All Rights Reserved \\[10mm]
    \end{copyrightpage}
  \fi
}%
%
%    \end{macrocode}
% \end{macro}
% \begin{macro}{\makepublicdomain} Or, if chosen, |\makepublicdomain| macro
% creates a copyright page (using earlier |copyrightpage| environment) that
% puts the document in public domain.
%    \begin{macrocode}
\newcommand{\makepublicdomain}{%
  \ifdiss@final
    \begin{copyrightpage}
       This document is in the public domain.
    \end{copyrightpage}
  \fi
}%
%
%    \end{macrocode}
% \end{macro}
% Define some new name macros and redefine other name macros as below. These
% are the names of the respective sections in your dissertation document. If
% there's a need to change any name, you must use a similar command in the
% preamble of your document.
%    \begin{macrocode}
\providecommand{\abstractname}{Abstract}
\providecommand{\dedicationname}{\mbox{}}
\providecommand{\prefacename}{PREFACE}
\providecommand{\acknowledgename}{ACKNOWLEDGMENTS}
\providecommand{\symbolsname}{SYMBOLS}
\renewcommand{\tablename}{TABLE}
\renewcommand{\figurename}{Figure}
\renewcommand{\chaptername}{CHAPTER}
\renewcommand{\appendixname}{APPENDIX}
\renewcommand{\contentsname}{CONTENTS}
\renewcommand{\listfigurename}{FIGURES}
\renewcommand{\listtablename}{TABLES}
\renewcommand{\bibname}{BIBLIOGRAPHY}
\renewcommand{\indexname}{INDEX}
%
%    \end{macrocode}
% \begin{environment}{abstract} This environment is adapted
% from the {\sffamily report}
% class since the {\sffamily book} class does not have one. Additionaly, we add
% a |\pdfbookmark| for the abstract in the pdf document.
%    \begin{macrocode}
\newenvironment{abstract}{%
  \ifpdf
     \pdfbookmark[0]{\abstractname}{abstract}%abstract.0
  \fi
  \typeout{Abstract page(s)}
  \renewcommand{\@oddfoot}{\@empty}
  \renewcommand{\@evenfoot}{\@empty}
%    \end{macrocode}
% If the abstract extends to a second page,
% place the author's name in top right corner of that page.  Make
% sure it's upright, as required by the University and that this appears
% at $0.75''$ from the top.
%    \begin{macrocode}
  \let\@evenhead\@oddhead
  \renewcommand{\@oddhead}{\hfil{\upshape\@author}}
  \titlepage
  \null
  \begin{center}
  \vspace*{0.25in}
  {\normalsize\mdseries \normalspacing
     \@title \\[3.5ex]
     \normalsize\abstractname \\ by \\ \@author\space}%
   \@endparpenalty \@M
   \end{center}\par}%
{\par\vfil\null\endtitlepage}
%
%    \end{macrocode}
% \end{environment}
% \begin{environment}{dedication} The ``dedication'' environment is similar to
% the ``abstract'' environment. This page is numbered 2 and the
% subsequent pages are numbered accordingly.
%    \begin{macrocode}
\newenvironment{dedication}{%
  \global\diss@dedicationtrue
  \typeout{Dedication page}
  \ifpdf
      \pdfbookmark[0]{\dedicationname}{dedication}%dedication.0
  \fi
  \chapter*{\dedicationname}%
  \thispagestyle{plain}
  \setcounter{page}{2}
  \null\centering}
{\par\null\clearpage}%
%
%    \end{macrocode}
% \end{environment}
% \begin{macro}{\tableofcontents} The |\tableofcontents| macro is redefined to
% begin at page 2 if the dedication environment does not exist. It is
% single-spaced.
%    \begin{macrocode}
\renewcommand\tableofcontents{%
  \ifdiss@dedication\relax\else\setcounter{page}{2}\fi
  \chapter*{\contentsname}%
  \ifpdf
      \pdfbookmark[0]{\contentsname}{contents}%contents.0
  \fi
  \singlespacing
  \@starttoc{toc}%
  \normalspacing
  }
%
%    \end{macrocode}
% \end{macro}
% \begin{macro}{\listoffigures}
% \begin{macro}{\listoftables}
% These macros are modified to add the
% |\listfigurename| and |\listoftables|
% to the Table of Contents. Both of these are also single spaced. The
% inter-entry spacing is changed by adding a |\vskip| after each entry. This is
% done in the |figure| and |table| environments later.
%    \begin{macrocode}
\renewcommand\listoffigures{%
    \chapter*{\listfigurename}%
    \addcontentsline{toc}{chapter}{\listfigurename}%
    \typeout{List of figures - \listfigurename}
    \singlespacing
    \@starttoc{lof}%
    \normalspacing
}
%
\renewcommand\listoftables{%
    \chapter*{\listtablename}%
    \addcontentsline{toc}{chapter}{\listtablename}%
    \typeout{List of tables - \listtablename}
    \singlespacing
    \@starttoc{lot}%
    \normalspacing
}
%
%    \end{macrocode}
% \end{macro}
% \end{macro}
% \begin{environment}{preface}
% \begin{environment}{acknowledgement}
% These environments are similar to the ``dedication'' environment. They are
% defined as |\chapter*{}| so they are not numbered and not added to Table of
% Contents and so, add that manually by using |\addcontentsline|.
%    \begin{macrocode}
\newenvironment{preface}{%
  \typeout{Preface page}
  \chapter*{\prefacename}
  \addcontentsline{toc}{chapter}{\prefacename}%
}%
{\par\null\clearpage}%
%
\newenvironment{acknowledge}{%
  \typeout{Acknowledgment page}
  \chapter*{\acknowledgename}
  \addcontentsline{toc}{chapter}{\acknowledgename}%
}%
{\par\null\clearpage}%
%
%    \end{macrocode}
% \end{environment}
% \end{environment}
% \begin{environment}{symbols}
% \begin{macro}{\sym}
% Define \textsf{symbols} environment which lays out it as a |\chapter*| and
% adds |\symbolsname| to the TOC. The environment is actually a horizontally
% centered \textsf{longtable} environment. To aid entry of {\em symbol} and its
% definition, |\sym| macro command is also defined.
%    \begin{macrocode}
\newcommand{\sym}[2]{\ensuremath{#1} & #2 \\}
\newenvironment{symbols}[1][rl]{%
  \typeout{Symbols page}
  \chapter*{\symbolsname}%
  \addcontentsline{toc}{chapter}{\symbolsname}%
  \begin{center}\begin{longtable}{#1}}%
{\end{longtable}\end{center}\par\null}
%
%    \end{macrocode}
% \end{macro}
% \end{environment}
% Modify chapter definition in |\@chapter| to put the word
% ``Chapter''  (|\@chapapp|) in the Table of Contents.
% That is, now the TOC will contain \par�
% ``\textbf{Chapter 1:  First chapter}''  rather than
% ``\textbf{1.  First chapter}'' \newline�
% The rest of the format code is essentially similar to
% that in the \textsf{book} class.
%    \begin{macrocode}
\def\@chapter[#1]#2{
  \ifnum \c@secnumdepth >\m@ne
  \if@mainmatter
    \refstepcounter{chapter}%
    \typeout{\@chapapp\space\thechapter.}%
    \addcontentsline{toc}{chapter}%
      {{\@chapapp\ \thechapter: #1}}%
    \else
      \addcontentsline{toc}{chapter}{#1}%
    \fi
  \else
     \addcontentsline{toc}{chapter}{#1}%
  \fi
  \chaptermark{#1}%
  \addtocontents{lof}{\protect\addvspace{10\p@}}%
  \addtocontents{lot}{\protect\addvspace{10\p@}}%
  \@makechapterhead{#2}%
  \@afterheading }%
%
%    \end{macrocode}
% Now format section headings to conform to the official guidelines.
% \begin{macro}{\@makechapterhead}
% First, modify the chapter heading label to be normalsize'd
% and centered. Instead of the bold-faced heading label,
% also make it |\mdseries|.
% If we are in the |\mainmatter|, we add ``CHAPTER''
% and chapter number before actually putting the chapter name otherwise only the
% ``chapter name'' is put. Note that chapter/section headings must all be
% double-spaced.
%    \begin{macrocode}
\renewcommand{\@makechapterhead}[1]{%
  \vspace*{.625in}%
  {\parindent \z@ \raggedright
    \ifnum \c@secnumdepth >\m@ne
      \normalfont\normalsize%
      \if@mainmatter
        \ifcentered@chaptitle\center\else\relax\fi%
        \@chapapp{} \thechapter\par\nobreak
      \fi
    \fi
    \interlinepenalty\@M
    \ifcentered@chaptitle\center\else\relax\fi%
    \mdseries{#1}\par\nobreak
    \vskip 30\p@
  }}
%
%    \end{macrocode}
% \end{macro}
% \begin{macro}{\@makeschapterhead}
% Make the TOC, LOF, LOT and other |\chapter*| headings in normal size,
% and |\mdseries| by modifying the macro |\@makeschapterhead|. Although these
% heading labels are usually fitting in single-line, we copy the formatting for
% the chapter heading label (single-spacing) and make the spacing double again
% for the text.
%    \begin{macrocode}
\renewcommand{\@makeschapterhead}[1]{%
  \vspace*{.625in}%
  {\parindent \z@ \raggedright
     \normalfont\normalsize%
     \interlinepenalty\@M
     \ifcentered@chaptitle\center\else\relax\fi
     \mdseries{#1}\par\nobreak
     \vskip 30\p@
  }}
%
%    \end{macrocode}
% \end{macro}
% Now, set the section labels to |\mdseries| rather than bold-faced. We also
% make sure that these are set in normal spacing, font and size.
% This is done for each
% of |\section|, |\subsection|, |\subsubsection|, |\subsubsubsection|,
% |\paragraph| and |\subparagraph|.
%    \begin{macrocode}
\renewcommand\section{\suppressfloats[t]%
    \@startsection {section}{1}{\z@}%
    {-4.2ex \@plus -1ex \@minus -.2ex}%
    {1.8ex \@plus.2ex}%
    {\normalfont\normalsize\mdseries} }
\renewcommand\subsection{\suppressfloats[t]%
    \@startsection{subsection}{2}{\z@}%
    {-3.9ex\@plus -1ex \@minus -.2ex}%
    {1.2ex \@plus .2ex}%
    {\normalfont\normalsize\mdseries} }
\renewcommand\subsubsection{\suppressfloats[t]%
    \@startsection{subsubsection}{3}{\z@}%
    {-3.9ex\@plus -1ex \@minus -.2ex}%
    {1.2ex \@plus .2ex}%
    {\normalfont\normalsize\mdseries} }
\renewcommand\paragraph{%
    \@startsection{paragraph}{4}{\z@}%
    {3.9ex \@plus1ex \@minus.2ex}%
    {-1em}%
    {\normalfont\normalsize\mdseries} }
\renewcommand\subparagraph{%
    \@startsection{subparagraph}{5}{\parindent}%
    {3.9ex \@plus1ex \@minus .2ex}%
    {-1em}%
    {\normalfont\normalsize\mdseries} }
%
%    \end{macrocode}
% \begin{macro}{\l@chapter}
% Modify the macro |\l@chapter| that formats chapter titles in the
% contents-like files (|.toc|, |.lof| and |.lot|) by adding a |\@dottedtocline|
% macro. The indent width is set to 1.5em - to line up a continued line with
% the section number below it. We also leave lesser space between each chapter
% and the last section entry than the default.
%    \begin{macrocode}
\renewcommand*{\l@chapter}[2]{%
  \addpenalty{-\@highpenalty}%
  \setlength\@tempdima{1.5em}%
  \begingroup \leavevmode
  \@dottedtocline{1}{0pt}{\@tempdima}{#1}{#2}
  \par
  \penalty\@highpenalty
  \endgroup
}
%
%    \end{macrocode}
% \end{macro}
% \begin{macro}{\l@subsubsection}
% The macro |\l@subsubsection| is modified to indent the |\subsubsection| label
% same as that for |\subsection| in the table of contents. We also increase the
% number of section-depth by 1 and force subsubsection entry in the TOC
% by increasing the |\tocdepth|. In addition, the label number of
% |\subsubsection| is defined to be similar to that for |\subsection| i.\/e.\/
% all arabic numerals.
%    \begin{macrocode}
\renewcommand{\l@subsubsection}{%
   \@dottedtocline{3}{3.8em}{4.1em}}
\addtocounter{secnumdepth}{1}
\addtocounter{tocdepth}{1}
\renewcommand{\thesubsubsection}{%
   \thesubsection.\arabic{subsubsection}}
%
%    \end{macrocode}
% \end{macro}
% \begin{environment}{quote}
% Redefine the |quote| environment to be single-spaced instead of being same
% as the rest of the text.
%    \begin{macrocode}
\renewenvironment{quote}
               {\list{}{\rightmargin\leftmargin}%
                 \singlespacing
               \item\relax}
               {\endlist}
%
%    \end{macrocode}
% \end{environment}
% Set some lengths that are used in the |table| and the |figure|
% environments. Note that we set the caption width (|\capwidth|)
% to be 90\% of the |\textwidth|.
%    \begin{macrocode}
\setlength\abovecaptionskip{20\p@}
\newlength\capwidth
\setlength{\capwidth}{0.90\textwidth}
\newlength\abovetableskip
\newlength\belowtableskip
\newlength\abovefigureskip
\newlength\belowfigureskip
\setlength\abovetableskip\belowcaptionskip
\setlength\belowtableskip\abovecaptionskip
\setlength\abovefigureskip\abovecaptionskip
\setlength\belowfigureskip\belowcaptionskip
%
%    \end{macrocode}
% \begin{environment}{figure}
% For the |figure| environment, first some skip ``lengths'' are set, then
% use |\@makefigurecaption| to
% format the captions instead of the default |\@makecaption|, since the
% layout is different for |figure| and the |table| environment. Further
% add a |\vskip| to each entry in |.lof| file so that the
% inter-caption spacing seems double-spaced.
%    \begin{macrocode}
\renewenvironment{figure}{%
   \setlength{\abovecaptionskip}{\abovefigureskip}
   \setlength{\belowcaptionskip}{\belowfigureskip}
   \let\@makecaption\@makefigurecaption
   \@float{figure}}%
   {%
   \addtocontents{lof}{ {\vskip 0.4em} }%
   \end@float%
}
%
%    \end{macrocode}
% \end{environment}
% \begin{macro}{\@makefigurecaption}
% The |\@makefigurecaption| is defined to format the caption in a parbox with
% width equal to |\capwidth| and is formatted in
% single-spacing. The interline-spacing is then changed to double after
% the caption.
%    \begin{macrocode}
\long\def\@makefigurecaption#1#2{%
  \vskip\abovecaptionskip
  \begin{center}
  \parbox{\capwidth}{
    \centering\singlespacing
    {#1}. {#2}%\par
  \vskip\belowcaptionskip\normalspacing }%
  \end{center}
}%
%
%    \end{macrocode}
% \end{macro}
% \begin{environment}{table}
% After setting the above and below skip lengths,
% the |table| environment is set to be
% single spaced. However, to obtain double-spacing between the entries,
% redefine the |\arraystretch| to be equivalent to the
% |\double@baselinestretch|. This way, while there is double-spaced entries,
% the entry itself is single-spaced. Similar to that in
% |\@makefigurecaption|, a |\vskip| is added to each entry in the |.lot| file.
%    \begin{macrocode}
\renewenvironment{table}{%
   \setlength{\abovecaptionskip}{\abovetableskip}
   \setlength{\belowcaptionskip}{\belowtableskip}
   \singlespacing
   \renewcommand{\arraystretch}{\double@baselinestretch}
   \let\@makecaption\@maketablecaption
   \@float{table}}%
   {%
   \addtocontents{lot}{ {\vskip 0.4em} }%
   \end@float%
}
%
%    \end{macrocode}
% \end{environment}
% \begin{macro}{\@maketablecaption}
% The |\@maketablecaption| is defined similar to |\@makefigurecaption|
% to have the table label and caption in
% separate lines and with normal-spacing (double-spaced).
%    \begin{macrocode}
\long\def\@maketablecaption#1#2{
  \vskip\abovecaptionskip
  \begin{center}
    \parbox{\capwidth}{
    \centering\normalspacing
    {#1}\\[\single@skip]
    {#2}%\par
  \vskip\belowcaptionskip }%
  \end{center}
}
%
%    \end{macrocode}
% \end{macro}
% \begin{macro}{\longtable}
% Similar to the |table| environment,
% the |longtable| environment is made singly-spaced but the |\arraystretch| is
% made equal to double the |baselinestretch|.
%    \begin{macrocode}
\renewcommand\longtable{%
   \singlespacing
   \renewcommand{\arraystretch}{\double@baselinestretch}
   \begingroup
   \@ifnextchar[\LT@array{\LT@array[x]}}
%
%    \end{macrocode}
% \end{macro}
% \begin{macro}{\endlongtable}
% This bit is taken from the |longtable.sty|. In order to obtain double-spacing
% in the list of tables, a |\vskip| of 0.4em is added to |.lot| file.
%    \begin{macrocode}
\renewcommand\endlongtable{%
  \crcr
  \noalign{%
    \let\LT@entry\LT@entry@chop
    \xdef\LT@save@row{\LT@save@row}}%
  \LT@echunk
  \LT@start
  \unvbox\z@
  \LT@get@widths
  \if@filesw
    {\let\LT@entry\LT@entry@write\immediate\write\@auxout{%
      \gdef\expandafter\noexpand
        \csname LT@\romannumeral\c@LT@tables\endcsname
          {\LT@save@row}}}%
  \fi
  \ifx\LT@save@row\LT@@save@row
  \else
    \LT@warn{Column \@width s have changed\MessageBreak
             in table \thetable}%
    \LT@final@warn
  \fi
  \endgraf\penalty -\LT@end@pen
  \addtocontents{lot}{ {\vskip 0.4em} }%
  \endgroup
  \global\@mparbottom\z@
  \pagegoal\vsize
  \endgraf\penalty\z@\addvspace\LTpost
  \ifvoid\footins\else\insert\footins{}\fi
}
%
%    \end{macrocode}
% \end{macro}
% \begin{macro}{\LT@makecaption}
% For the |longtable| environment, the |\LTcapwidth| is set equal to
% |\capwidth|. In order to obtain consistent table captions, the command
% |\LT@makecaption| is modified in a similar manner as |\maketablecaption|.
%    \begin{macrocode}
\setlength{\LTcapwidth}{\capwidth}
\renewcommand\LT@makecaption[3]{%
  \LT@mcol\LT@cols c{\hbox to\z@{\hss\parbox[t]\LTcapwidth{%
  \vskip\abovetableskip%
    \centering\normalspacing
    #1{#2 }\\[\single@skip]
    {#3}\par
  \endgraf\vskip\belowtableskip}%
  \hss}}}
%
%    \end{macrocode}
% \end{macro}
% \begin{macro}{\timenow}
% This macro is used in making the |\draftheader| and |\reviewheader| below. It
% outputs time in |HH:MM| format.
%    \begin{macrocode}
\newcommand\timenow{%
  \@tempcnta=\time \divide\@tempcnta by 60 \number\@tempcnta:\multiply
  \@tempcnta by 60 \@tempcntb=\time \advance\@tempcntb by -\@tempcnta
  \ifnum\@tempcntb <10 0\number\@tempcntb\else\number\@tempcntb\fi}
%
%    \end{macrocode}
% \end{macro}
% \begin{macro}{\diss@header}
% This header is used in the dissertation document when the |draft| or |review|
% option is used. These headers serve as a note for the date and time of the
% document compilation.
%    \begin{macrocode}
\newcommand{\diss@header}{%
    \ifdiss@review Review \else Draft \fi document [\today\/ at \timenow\/]
    }%
%
%    \end{macrocode}
% \end{macro}
% The header prepared above is put in the document by modifiying the {\em
% plain} and {\em empty} pagestyles except when the |final| option is chosen.
%    \begin{macrocode}
\ifdiss@final
    \renewcommand{\ps@plain}{
        \renewcommand{\@oddhead}{\@empty}
        \renewcommand{\@oddfoot}{\hfil\thepage\hfil}
        \let\@evenhead\@oddhead
        \let\@evenfoot\@oddfoot
    }%
\else
    \renewcommand{\ps@plain}{
        \renewcommand{\@oddhead}{\framebox[\textwidth]{
           \centering\footnotesize\tt\diss@header}}%
        \renewcommand{\@oddfoot}{\hfil\textrm{\thepage}\hfil}
        \let\@evenhead\@oddhead
        \let\@evenfoot\@oddfoot
    }%
    \renewcommand{\ps@empty}{
        \renewcommand{\@oddhead}{\framebox[\textwidth]{
           \centering\footnotesize\tt\diss@header}}%
        \renewcommand{\@oddfoot}{\@empty}
        \let\@evenhead\@oddhead
        \let\@evenfoot\@oddfoot
    }%
\fi
%
%    \end{macrocode}
% \begin{macro}{\bibsection}
% By redefining |\bibsection| macro, add the |\bibname| to the table of
% contents and as a chapter heading for the bibliography.
%    \begin{macrocode}
\renewcommand{\bibsection}{
  \chapter*{\bibname}%
  \addcontentsline{toc}{chapter}{\bibname}%
}%
%
%    \end{macrocode}
% \end{macro}
% \begin{macro}{\bibfont}
% Changed the |\bibfont| macro to obtain single-spacing within each
% bibliographic entry. Between different entries, it is still |\normalspacing|.
% In addition, when |numrefs| option is selected, the |\@biblabel| is redefined
% to number the bibliographic entries as |1. xxxx| instead of the default
% |[1] xxxx|.
%    \begin{macrocode}
\newcommand{\bibfont}{\singlespacing}
\ifnum@refs
  \renewcommand{\@biblabel}[1]{\hfill#1.\hfill}
\fi
%
%    \end{macrocode}
% \end{macro}
% Lastly, after the bibliography in the final document,
% add a framed box which contains blurb about
% the typesetting program and \nddiss\space version used for preparing the
% dissertation document.
%    \begin{macrocode}
\ifdiss@final
\AtEndDocument{
    \vfill
    \centering\singlespacing
    \framebox[0.85\textwidth]{
    \begin{minipage}{0.80\textwidth}\footnotesize%
    \centering \itshape This document was prepared \& typeset with
    \upshape\ifpdf pdf\LaTeX\else\LaTeXe\fi\itshape , and
    formatted with \upshape\nddiss\xspace\itshape classfile
    (v\dissfileversion [\dissfiledate]) provided by Sameer Vijay.
    \end{minipage} }
    \clearpage}
\else\relax\fi
%
% \endinput
% End of file `nddiss2e.cls'.
%    \end{macrocode}
%%
% \iffalse
%</class>
% \fi
%
% \iffalse
%<*template>
\documentclass[draft]{nddiss2e}
                     % Available options are
                     % (a) draft + 10pt/11pt/12pt + twoadvisors + textrefs
                     % (b) review + noinfo + twoadvisors + textrefs
                     % (c) final + noinfo + twoadvisors + textrefs

\begin{document}

\frontmatter         % All the items before Chapter 1 go in ``frontmatter''

\title{ }            % Title of Work
\author{ }           % Author's name
\work{ }             % ``Dissertation'' or ``Thesis''
\degprior{ }         % All prior degrees
\degaward{ }         % Degree you're aiming for
\advisor{ }          % Advisor's name
 % \secondadvisor{ } % Second advisor, if used option ``twoadvisors''
\department{ }       % Name of the department

\maketitle           % The title page is created now

 % \copyrightholder{ } % If you're not the copyright holder
 % \copyrightyear{ }   % If the copyright is not for the current year
 % \makecopyright      % If not making your work public domain
                       % uncomment out \makecopyright
 % \makepublicdomain   % Uncomment this to make your work public domain

 % \begin{abstract}
 % \end{abstract}
 %                         % Either place the text between begin/end, or
 % \include{abstract}  % put it in a file to be included

 % \renewcommand{\dedicationname}{\mbox{}} % Replace \mbox{} if you want
                                           % something else
 % \begin{dedication}
 % \end{dedication}
 %                       % Use one of the two choices to add dedication text
 % \include{dedication}

\tableofcontents
\listoffigures
\listoftables

 %% \renewcommand{\symbolsname}{newsymname} % Replace ``newsymname'' with
                                           % the name you want, and uncomment
 % \begin{symbols}
 % \end{symbols}
 %                       % Use one of the two choices to add symbols text
 % \include{symbols}

 %% \renewcommand{\prefacename}{ } % If you want another Preface name, add
                                   % something else, and uncomment
 % \begin{preface}
 % \end{preface}
 %                       % Use one of the two choices to add preface text
 % \include{preface}

 %% \renewcommand{\acknowledgename}{ } % If you want another Acknowledgement name
                                       % add something else, and uncomment
 % \begin{acknowledge}
 % \end{acknowledge}
 %                       % Use one of the two choices to add acknowledge text
 % \include{acknowledgement}

\mainmatter
 % Place the text body here.
 % \include{chapter-one}

\appendix

 % If you have appendices, add them here.
 % Begin each one with \chapter{title} as before- the \appendix command takes
 % care of renaming chapter headings and creates a new page in the Table of
 % Contents for them.
 % \include{appendix-one}

\backmatter              % Place for bibliography and index

\bibliographystyle{nddiss2e}  % Use the ``nddiss2e'' biblio-style
 \bibliography{ }           % input the bib-database file name

\end{document}

%</template>
% \fi
%%
% \Finale
\endinput
%% End of file : nddiss2e.dtx
